

% les spécifications du programme :

%\section{Analyse statique}

%La vérification fonctionnelle d'un programme informatique, pour un logiciel ou un matériel, correspond à la vérification que l'implémentation du programme est conforme à ses spécifications  \footnote{Une spécification est un ensemble explicite d'exigences à satisfaire par un matériel, produit ou service} ainsi que la vérification de l'absence d'erreurs lors de leur exécution. La vérification fonctionnelle est une étape essentielle dans le développement d'un produit de sécurité dans lequel les fonctions de sécurité d'intégrité, de confidentialité et de disponibilité doivent être garanties.
%\\
%\noindent
%Un programme informatique peut être analysé à travers son code source ou en l'exécutant. L'analyse du code source, sans exécution du programme informatique, est une analyse statique, alors qu'exécuter un programme à des fins d'analyse, par exemple en réalisant des tests, est une analyse dynamique. De manière générale l'analyse statique doit analyser l'entièreté d'un code source tandis que l'analyse dynamique ne se concentre que sur l'exécution du programme avec un jeu de données spécifique.
%\\
%\noindent
%Il existe plusieurs méthodes d'analyse statique. Par exemple, la revue de code en faisant lire le code source du programme informatique par une personne extérieure à l'équipe de développement. Une autre manière d'analyser le code de manière statique consiste à l'utilisation d'outils automatiques permettant de vérifier formellement le code. C'est ce type d'analyse qui a été effectuée dans ce stage.

%L'analyse statique de programmes est l'utilisation d'outil comme lint et ses successeurs pour vérifier le respect d'un ensemble de règles de programmation. L’analyse statique permet d'automatiser (systématiser) certains contrôles (complexité du code, respect de règles de codage…) mais ne permet pas de vérifier certains points plus abstraits (réutilisabilité, efficience, respect du cahier des charges…) qui nécessitent une relecture manuelle.

%L'analyse statique et la revue par les pairs sont donc complémentaires, permettant d'étendre qualitativement et quantitativement la revue du code. L'outil d'analyse statique permet notamment de se décharger des vérifications de détails et favorise une vision plus globale du système étudié lors de la revue de code.

%Les premières idées d'analyse statique se trouvent dans les toutes premières recherches sur les ordinateurs naissants, à la fin des années 1940. Ce qui était cher à l'époque ce n'était pas tant le système construit que l'accès à la ressource (l'ordinateur). Ce qu'on appelle maintenant les preuves de programmes sont à l'évidence les premières traces d'analyse statique et on en trouve l'évidence dans la conférence d'Alan Turing2 en 1949, puis plus tard dans le travail sur les méthodes formelles dans les organigrammes de Robert Floyd3 et enfin dans la logique de Hoare.

%L’analyse statique englobe une famille de méthodes formelles qui dérivent automatiquement de l’information sur le comportement de logiciels ou de matériels informatiques.  – grosso modo, des événements qui poussent le programme à « planter ». Parmi les erreurs les plus courantes en ce genre, on peut citer les fautes de frappe pures et simples, notamment pour les langages de programmation sensibles à la casse, les formules faisant intervenir des variables non initialisées ou même non-déclarées, les références circulaires, l'emploi de syntaxes non-portables.

%L’analyse de programmes (y compris la recherche d'erreurs possibles à l'exécution) n'est pas déterministe : il n’existe aucune méthode « mécanique » qui peut toujours dire sans se tromper au vu d’un programme si celui-ci va ou non produire des erreurs à l’exécution. C’est là un résultat mathématique fondé sur des résultats d'Alonzo Church, Kurt Gödel et Alan Turing dans les années 1930 (voir le problème de l'arrêt et le théorème de Rice).

%\section{La vérification formelle}

La vérification formelle consiste à raisonner rigoureusement, à l'aide de logique mathématique, sur des programmes informatiques afin de démontrer leur validité par rapport à une certaine propriété. Un exemple de propriété est l'absence de Run Time Error \footnote{Une RTE correspond à une erreur survenue lors de l'exécution d'un programme, pouvant provoquer son arrêt non prévu mais aussi pouvant mener à un comportement indéfini du programme} (RTE dans le reste du document)
\\
Cette vérification est basée sur la sémantique des programmes, c'est-à-dire sur le lien entre le langage signifiant (le langage de programmation) et le langage signifié (description mathématique du programme sous forme logique, d'automates ou autre).
Les mathématiques sont donc utilisées pour concevoir, réaliser ou vérifier un système informatique (spécification/conception/vérification formelle). La vérification formelle présente l'avantage d'être précise et non ambiguë (toutes les propriétés initialement exprimées en langage naturel sont décrites de manière mathématique).
Ainsi, l'objectif de vérifier formellement un programme (ou de le prouver) est de s’assurer que, quelle que soit l’entrée fournie au programme, si elle respecte la spécification, alors le programme fera ce qui est attendu. Les mathématiques pourront prouver que le programme ne peut avoir que les comportements qui sont spécifiés et que les erreurs d’exécution n’en font pas partie.
\newline
\newline
\noindent
La vérification formelle peut être illustrée en raisonnant sur l'identité remarquable suivante : $ (a + b)^2 = a^2 + b^2 + 2ab $
\\
\noindent Il est possible de tester cette équation avec plusieurs valeurs pour vérifier l'égalité entre les deux termes, mais cette vérification ne peut pas être exhaustive ou cela nécéssiterait de tester l'équation sur $2^{31}*2^{31}$ valeurs. Il est également possible de prouver ce résultat simplement en développant et en factorisant pour montrer l’équivalence. Il s'agit dans ce cas d'une vérification formelle.
%La vérification peut être partielle ou non, formelle ou non.
\newline
\newline
\noindent
De manière générale, la vérification formelle comprend plusieurs étapes:
\begin{itemize}
	\item la spécification \footnote{Une spécification est une description d'exigences à satisfaire par un matériel, un produit ou un service} formelle du programme informatique en utilisant les mathématiques, qui consiste au passage d'un langage informatique à un langage mathématique ;
	\item la vérification du respect de certaines proprités de la spécification : propriétés fonctionnelles, de sûreté et de sécurité, par exemple l'absence de RTE.
\end{itemize}

Deux notions importantes permettent par ailleurs de caractériser la vérification formelle:
\begin{itemize}
	\item La complétude : la vérification formelle ne comporte pas de faux positifs, chaque erreur détectée est une véritable erreur (sous approximation du code analysé). Cela ne permet toutefois pas de conclure à l'absence d'erreur dans le code ;
	\item La correction : la vérification formelle ne comporte pas de faux négatifs : toutes les erreurs potentielles sont détectées (sur approximation du code analysé). Néanmoins, des faux positifs peuvent être relevés lors de la vérification formelle, ce qui nécessite un travail d'analyse supplémentaire.
\end{itemize}

%soundness + correctness : pas de faux négatifs

% l’objectif est le même : s’assurer de la fiabilité, de la sécurité ainsi que de l’absence de bugs informatiques d’un système par des preuves mathématiques.

% Vérification d'un logiciel : bien fait
% validation d'un logiciel : le bon