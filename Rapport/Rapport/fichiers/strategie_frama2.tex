%\documentclass{article}
%\usepackage[utf8]{inputenc}
%\usepackage{tikz}
%\usetikzlibrary{shapes.geometric, arrows}
%
\tikzstyle{startstop} = [rectangle, rounded corners, minimum width=3cm, minimum height=1cm,text centered, draw=black, fill=red!30]
\tikzstyle{io} = [trapezium, trapezium left angle=70, trapezium right angle=110, minimum width=3cm, minimum height=1cm, text centered, draw=black, fill=blue!30]
\tikzstyle{greffon} = [rectangle, minimum width=3cm, minimum height=1cm, text centered, text width=3cm, draw=black, fill=orange!30]
\tikzstyle{process} = [rectangle, minimum width=3cm, minimum height=1cm, text centered, text width=3cm, draw=black, fill=purple!30]
\tikzstyle{decision} = [diamond, minimum width=3cm, minimum height=1cm, text centered, draw=black, fill=green!30]
\tikzstyle{arrow} = [thick,->,>=stealth]
%
%\begin{document}
\begin{figure}[ht]
\centering
\begin{tikzpicture}[node distance=2cm]

\node (start) [startstop] {Code source à analyser};
%\node (in1) [io, below of=start] {Input};
\node (pro1) [process, below of=start] {Parsing};
\node (dec1) [decision, below of=pro1, yshift=-0.5cm] {Parsing OK ?};
\node (pro2) [greffon, below of=dec1, yshift=-0.5cm] {EVA};
\node (pro1a) [process, right of=dec1, xshift=2cm] {Ajouts des fichiers manquants};
\node (dec2a) [decision, below of=pro2, yshift=-0.1cm] {RTE ?};
\node (pro2a) [process, right of=dec2a, xshift=5cm] {Correction du code source / modification des options d'EVA};
\node (pro3) [greffon, below of=dec2a, yshift=-0.1cm] {WP};
\node (dec3) [decision, below of=pro3, yshift=-0.5cm] {Preuves OK ?};
\node (pro3a) [process, right of=dec3, xshift=8cm] {Correction du code source / modification des spécifications / modification des options de WP};
\node (stop) [startstop, below of=dec3, yshift=-0.5cm] {Victoire! Code formellement vérifié};

\draw [arrow] (start) -- (pro1);
%\draw [arrow] (in1) -- (pro1);
\draw [arrow] (pro1) -- (dec1);
\draw [arrow] (dec1) -- node[anchor=east] {oui} (pro2);
\draw [arrow] (dec1) -- node[anchor=south] {non} (pro1a);
\draw [arrow] (pro1a) |- (start);
\draw [arrow] (pro2) -- (dec2a);
%\draw [arrow] (dec2a) -- (dec2b);
%\draw [arrow] (dec2b) -- (dec2c);
\draw [arrow] (dec2a) -- node[anchor=south] {oui} (pro2a);
\draw [arrow] (dec2a) -- node[anchor=east] {non} (pro3);
\draw [arrow] (pro2a) |- (start);
\draw [arrow] (pro3) -- (dec3);
\draw [arrow] (dec3) -- node[anchor=south] {non} (pro3a);
\draw [arrow] (dec3) -- node[anchor=east] {oui} (stop);
\draw [arrow] (pro3a) |- (start);

\end{tikzpicture}

\caption{Stratégie d'utilisation de Frama-C}
\end{figure}

%\end{document}